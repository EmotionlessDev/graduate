% !TEX root = ../main.tex
\section{Постановка задачи}
\addcontentsline{toc}{section}{Постановка задачи}

\subsection{Цель работы}
Целью данного курсового проекта является разработка программного обеспечения для поддержки принятия решений в покере в реальном времени. Программа анализирует текущую игровую ситуацию и подсвечивает оптимальное действие игрока на префлопе (бет, колл, фолд, рейз) на основе заранее настроенных диапазонов рук.

Пользователь самостоятельно загружает и настраивает диапазоны для различных ситуаций и типов игр (кэш, MTT), включая:
\begin{itemize}
    \item RFI (Raise First In),
    \item BB DEF (Big Blind Defense),
    \item SB DEF (Small Blind Defense),
    \item ISO (Isolate),
    \item 3BET,
    \item 3BET DEF OOP.
\end{itemize}

\subsection{Описание задачи и архитектура системы}
Система состоит из следующих основных модулей:

\begin{enumerate}
    \item \textbf{Модуль загрузки и хранения диапазонов:} отвечает за импорт диапазонов пользователя из файлов (CSV, JSON или других форматов) и хранение их в удобной внутренней структуре.
    
    \item \textbf{Парсер игровой ситуации:} считывает состояние игры через OCR или API (если доступно) и определяет:
    \begin{itemize}
        \item позиции игроков,
        \item размеры стеков,
        \item текущие ставки,
        \item карты на столе.
    \end{itemize}
    
    \item \textbf{Модуль принятия решения:} сравнивает текущую игровую ситуацию с загруженными диапазонами и определяет оптимальное действие игрока:
    \begin{itemize}
        \item бет,
        \item колл,
        \item фолд,
        \item рейз.
    \end{itemize}
    Модуль должен быстро вычислять решение в реальном времени и выдавать его для отображения пользователю.
    
    \item \textbf{Модуль пользовательского интерфейса (UI):} отображает подсказку игроку о правильном действии. Возможные способы визуализации:
    \begin{itemize}
        \item текстовое сообщение поверх окна покер-рума, показывающее рекомендуемое действие,
        \item подсветка соответствующей кнопки (бет, колл, фолд, рейз) в интерфейсе покер-рума.
    \end{itemize}
    UI должен быть интуитивно понятным и не мешать основной игре.
\end{enumerate}

\subsection{Наиболее сложные моменты реализации}
При реализации проекта выделяются следующие трудности:

\begin{enumerate}
    \item \textbf{Реальное время:} программа должна быстро обрабатывать состояние игры и мгновенно подсвечивать действия, чтобы быть полезной игроку.
    
    \item \textbf{Парсинг игровой ситуации:} без официального API придётся использовать OCR для определения ставок, карт и позиций. Это требует точного распознавания информации на экране.
    
    \item \textbf{Совместимость с покер-румом:} программа не должна мешать работе клиента покерного рума.
    
    \item \textbf{UI и отображение подсказки:} необходимо выбрать удобный способ информирования игрока, который будет хорошо виден, но не перекрывать важные элементы интерфейса покер-рума.
\end{enumerate}
